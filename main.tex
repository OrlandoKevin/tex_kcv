\documentclass{article}
\usepackage[utf8]{inputenc}
\usepackage{graphicx}
\usepackage{booktabs}
\usepackage{caption}
\usepackage{float}

\title{Exemple d'Article LaTeX}
\author{Jean Dupont}
\date{24 mai 2025}

\begin{document}

\maketitle

\begin{abstract}
Cet article présente un exemple complet de document LaTeX avec titres, paragraphes, tableaux et graphiques.
\end{abstract}

\section{Introduction}

LaTeX est un système de composition de documents très utilisé pour la rédaction scientifique. Voici un exemple d'article structuré.

\section{Méthodologie}

Nous présentons ici la méthodologie utilisée pour générer les données et les graphiques.

\subsection{Données}

Les données utilisées sont fictives et servent d'exemple.

\section{Résultats}

\subsection{Tableau de résultats}

\begin{table}[H]
    \centering
    \caption{Exemple de tableau de résultats}
    \begin{tabular}{lcc}
        \toprule
        Catégorie & Valeur 1 & Valeur 2 \\
        \midrule
        A & 10 & 20 \\
        B & 15 & 25 \\
        C & 12 & 22 \\
        \bottomrule
    \end{tabular}
\end{table}

\subsection{Graphique}

\begin{figure}[H]
    \centering
    \includegraphics[width=0.6\textwidth]{example-image-a}
    \caption{Exemple de graphique (remplacer par votre image)}
\end{figure}

\section{Discussion}

Les résultats présentés montrent la structure d'un article scientifique type.

\section{Conclusion}

Cet exemple illustre l'utilisation de LaTeX pour la rédaction d'articles avec des éléments variés.

\end{document}
