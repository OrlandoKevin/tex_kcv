\documentclass[10pt, letterpaper]{article}

% Packages:
\usepackage[
  ignoreheadfoot, % set margins without considering header and footer
  top=2 cm, % seperation between body and page edge from the top
  bottom=2 cm, % seperation between body and page edge from the bottom
  left=2 cm, % seperation between body and page edge from the left
  right=2 cm, % seperation between body and page edge from the right
  footskip=1.0 cm, % seperation between body and footer
  % showframe % for debugging 
]{geometry} % for adjusting page geometry
\usepackage{titlesec} % for customizing section titles
\usepackage{tabularx} % for making tables with fixed width columns
\usepackage{array} % tabularx requires this
\usepackage[dvipsnames]{xcolor} % for coloring text
\definecolor{primaryColor}{RGB}{0, 79, 144} % define primary color
\usepackage{enumitem} % for customizing lists
\usepackage{fontawesome5} % for using icons
\usepackage{amsmath} % for math
\usepackage[
  pdftitle={CV - korlando},
  pdfauthor={Kevin Bosirany ORLANDO},
  pdfcreator={LaTeX with RenderCV},
  colorlinks=true,
  urlcolor=primaryColor
]{hyperref} % for links, metadata and bookmarks
\usepackage[pscoord]{eso-pic} % for floating text on the page
\usepackage{calc} % for calculating lengths
\usepackage{bookmark} % for bookmarks
\usepackage{lastpage} % for getting the total number of pages
\usepackage{changepage} % for one column entries (adjustwidth environment)
\usepackage{paracol} % for two and three column entries
\usepackage{ifthen} % for conditional statements
\usepackage{needspace} % for avoiding page brake right after the section title
\usepackage{iftex} % check if engine is pdflatex, xetex or luatex

% Ensure that generate pdf is machine readable/ATS parsable:
\ifPDFTeX
  \input{glyphtounicode}
  \pdfgentounicode=1
  % \usepackage[T1]{fontenc} % this breaks sb2nov
  \usepackage[utf8]{inputenc}
  \usepackage{lmodern}
\fi

% Some settings:
\AtBeginEnvironment{adjustwidth}{\partopsep0pt} % remove space before adjustwidth environment
\pagestyle{empty} % no header or footer
\setcounter{secnumdepth}{0} % no section numbering
\setlength{\parindent}{0pt} % no indentation
\setlength{\topskip}{0pt} % no top skip
\setlength{\columnsep}{0cm} % set column seperation
\makeatletter
\let\ps@customFooterStyle\ps@plain % Copy the plain style to customFooterStyle
\patchcmd{\ps@customFooterStyle}{\thepage}{
%  \color{gray}\textit{\small Kevin Bosirany ORLANDO - Page \thepage{} of \pageref*{LastPage}}
}{}{} % replace number by desired string
\makeatother
\pagestyle{customFooterStyle}

\titleformat{\section}{\needspace{4\baselineskip}\bfseries\large}{}{0pt}{}[\vspace{1pt}\titlerule]

\titlespacing{\section}{
  % left space:
  -1pt
}{
  % top space:
  0.3 cm
}{
  % bottom space:
  0.2 cm
} % section title spacing

\renewcommand\labelitemi{$\circ$} % custom bullet points
\newenvironment{highlights}{
  \begin{itemize}[
    topsep=0.10 cm,
    parsep=0.10 cm,
    partopsep=0pt,
    itemsep=0pt,
    leftmargin=0.4 cm + 10pt
  ]
}{
  \end{itemize}
} % new environment for highlights

\newenvironment{highlightsforbulletentries}{
  \begin{itemize}[
    topsep=0.10 cm,
    parsep=0.10 cm,
    partopsep=0pt,
    itemsep=0pt,
    leftmargin=10pt
  ]
}{
  \end{itemize}
} % new environment for highlights for bullet entries


\newenvironment{onecolentry}{
  \begin{adjustwidth}{
    0.2 cm + 0.00001 cm
  }{
    0.2 cm + 0.00001 cm
  }
}{
  \end{adjustwidth}
} % new environment for one column entries

\newenvironment{twocolentry}[2][]{
  \onecolentry
  \def\secondColumn{#2}
  \setcolumnwidth{\fill, 4.5 cm}
  \begin{paracol}{2}
}{
  \switchcolumn \raggedleft \secondColumn
  \end{paracol}
  \endonecolentry
} % new environment for two column entries

\newenvironment{header}{
  \setlength{\topsep}{0pt}\par\kern\topsep\centering\linespread{1.5}
}{
  \par\kern\topsep
} % new environment for the header

\newcommand{\placelastupdatedtext}{% \placetextbox{<horizontal pos>}{<vertical pos>}{<stuff>}
  \AddToShipoutPictureFG*{% Add <stuff> to current page foreground
  \put(
    \LenToUnit{\paperwidth-2 cm-0.2 cm+0.05cm},
    \LenToUnit{\paperheight-1.0 cm}
  ){\vtop{{\null}\makebox[0pt][c]{
    %\small\color{gray}\textit{Dernière mise à jour de Avril 2025}\hspace{\widthof{Last updated in September 2024}}
  }}}%
  }%
}%

% save the original href command in a new command:
\let\hrefWithoutArrow\href

% new command for external links:
\renewcommand{\href}[2]{\hrefWithoutArrow{#1}{\ifthenelse{\equal{#2}{}}{ }{#2 }\raisebox{.15ex}{\footnotesize \faExternalLink*}}}

%%%%%%%%%%%%%%%%%%%%%%%%%%%%%%%%%%%%%%%%%%%%%%%%%%%%%%%%%%%%%%%%%%%%%%%%%%%%%%%%

\begin{document}

\newcommand{\AND}{\unskip
  \cleaders\copy\ANDbox\hskip\wd\ANDbox
  \ignorespaces
}
\newsavebox\ANDbox
\sbox\ANDbox{}

\placelastupdatedtext
\begin{header}
  \textbf{\fontsize{20 pt}{20 pt}\selectfont Kevin Bosirany ORLANDO}

  \vspace{0.3 cm}

  \normalsize
  \mbox{{\color{black}\footnotesize\faMapMarker*}\hspace*{0.13cm}Montpellier, France}%
  \kern 0.25 cm%
  \AND%
  \kern 0.25 cm%
  \mbox{\hrefWithoutArrow{mailto:kevinbosirany@gmail.com}{\color{black}{\footnotesize\faEnvelope[regular]}\hspace*{0.13cm}kevinbosirany@gmail.com}}%
  \kern 0.25 cm%
  \AND%
  \kern 0.25 cm%
  \mbox{\hrefWithoutArrow{tel:+33-762-204-28-40}{\color{black}{\footnotesize\faPhone*}\hspace*{0.13cm}+33 (0)7 62 04 28 40}}%
  \kern 0.25 cm%
  \AND%
  \kern 0.25 cm%
  \mbox{\hrefWithoutArrow{https://linkedin.com/in/kevin-bosirany-orlando}{\color{black}{\footnotesize\faLinkedinIn}\hspace*{0.13cm}kevin-bosirany-orlando}}%
  \kern 0.25 cm%
  \AND%
  \kern 0.25 cm%
  \mbox{\hrefWithoutArrow{https://github.com/OrlandoKevin}{\color{black}{\footnotesize\faGithub}\hspace*{0.13cm}OrlandoKevin}}%
\end{header}

\vspace{0.3 cm - 0.3 cm}

% Profil
\section{Profil}

\begin{onecolentry}
  Ingénieur génie rural spécialisé dans la gestion de l'eau, avec deux ans d'
  expérience en modélisation hydrologique, hydraulique et agronomique.
  Compétences en programmation scientifique, en modélisation des systèmes
  hydrologiques complexes et en analyse de données.
  Autonome, rigoureux et passionné par la recherche et l'innovation dans le
  domaine de l'eau et de l'agriculture.
  Recherche un poste dans la gestion de l'eau, à travers des études
  hydrologiques, hydrauliques ainsi que l'adaptation face aux changements
  globaux.
\end{onecolentry}

% Expériences professionelles
\section{Expériences professionnelles}

% INRAE -IE
\begin{twocolentry}
{
  \textit{Montpellier, France}
  
  \textit{Sept 2023 – Aujourd'hui}
} {
  \textbf{Ingénieur d'études}

  Chargé de modélisation hydrologique et agronomique

  \textit{INRAE - UMR G-EAU / UMR Innovation}
}
\end{twocolentry}

\vspace{0.10 cm}

\begin{onecolentry}
  \begin{highlights}
    \item{
      Modèle Optirrig (Génération, analyse et optimisation des scénarios d'
      irrigation des cultures)
      \begin{itemize}
        \item Développement et maintenance du package R du modèle (OptirrigCORE)
        \item Développement et implémentation des formalismes du modèle
        (développement des plantes, bilan hydrique, etc.)
      \end{itemize}
    }
    \item{
      Projet JOICE (Joint Optimization of Irrigation and Control of irrigation
      induced Erosion)
      \begin{itemize}
        \item Étude du couplage des modèles Optirrig et Mahleran (Model for
        Assessing Hillslope-Landscape Erosion, Runoff and Nutrients)
      \end{itemize}
    }
    \item{
      Projet TALANOA-WATER
      \begin{itemize}
        \item Modélisation des règles de gestion des infrastructures
        hydrauliques (réservoirs, canaux, etc.)
        \item Initiation du couplage du modèle intégré hydro-agro-économique
        TALANOA
      \end{itemize}
    }
  \end{highlights}
\end{onecolentry}

\vspace{0.2 cm}

% INRAE
\begin{twocolentry} {
  \textit{Montpellier, France}  
  
  \textit{Mars – Août 2023}
} {
  \textbf{Stagiaire}

  Stage de fin d'études
  
  \textit{INRAE - UMR G-EAU / UMR Innovation - Projet TALANOA-WATER}
}
\end{twocolentry}

\vspace{0.10 cm}

\begin{onecolentry}
  \begin{highlights}
    \item Modélisation du réseau hydrographique et hydraulique du bassin
    versant de l'Aude
    \item Calage du modèle
    \item Etude d'impact du changement climatique sur les ressources en eau
    (Volumes Prélevables)
  \end{highlights}
\end{onecolentry}

\vspace{0.2 cm}

% IAV
\begin{twocolentry}
  {
    \textit{Rabat, Maroc}

    \textit{Fév - Juil 2022}
  }
  {
    \textbf{Stagiaire}

    Stage de fin d'études

    \textit{Institut Agronomique et Vétérinaire Hassan II - Département
    Génie Rural}
  }
\end{twocolentry}

\vspace{0.10 cm}

\begin{onecolentry}
  \begin{highlights}
    \item Détermination de la consommation et de la productivité de l'eau
    dans le bassin de l'Oum Er Rbia
    \item Acquisition, traitement et analyse des données issues de la
    télédétection (FAO WaPOR)
  \end{highlights}
\end{onecolentry}

\vspace{0.2 cm}

% ADI
\begin{twocolentry}
  {
    \textit{Rabat, Maroc}

    \textit{Juil - Sept 2021}
  }
  {
    \textbf{Stagiaire}

    Stage d'insertion en milieu professionnel

    \textit{ADI (Compagnie d'Aménagement et de Développement Industriel)}
  }
\end{twocolentry}

\vspace{0.10 cm}

\begin{onecolentry}
  \begin{highlights}
    \item Dimensionnement du réseau projeté pour la reconversion en irrigation
    localisée du secteur RD1 du périmètre de la Tessaout aval
  \end{highlights}
\end{onecolentry}
  
% Parcours académiques
\section{Parcours académiques}
% EA
\begin{twocolentry}
  {
  \textit{Montpellier, France}

  \textit{Sept 2022 – Sept 2023}
  }
  {
    \textbf{Master 2 Science de l'Eau - Eau et Agriculture}

    \textit{Université de Montpellier - Institut Agro Montpellier - AgroParisTech}
  }
\end{twocolentry}

\vspace{0.2 cm}

% GR
\begin{twocolentry}
  {
    \textit{Rabat, Maroc}

    \textit{Sept 2019 – Juil 2022}
  }
  {
    \textbf{Ingénieur Génie Rural - Eau, Environnement et Infrastructures}

    \textit{Institut Agronomique et Vétérinaire Hassan II}
  }
\end{twocolentry}

\vspace{0.2 cm}

% Hydraulique

\begin{twocolentry}
  {
    \textit{Antananarivo, Madagascar}

    \textit{}
  }
  {
    \textbf{Licence Génie Civil - Hydraulique et Aménagement}

    \textit{Ecole Supérieure Polytechnique d'Antananarivo}
  }
\end{twocolentry}

% Compétences
\section{Compétences}
  %  
  \begin{onecolentry}
    \textbf{Langage de programmation:} R, Python, Fortran
  \end{onecolentry}

  \vspace{0.2 cm}

  \begin{onecolentry}
    \textbf{Versioning:} Git, Gitlab, Github, CI/CD
  \end{onecolentry}

  \vspace{0.2 cm}

  \begin{onecolentry}
    \textbf{Logiciels:} Microsoft Office, CAO/DAO, Epanet, PVSyst, Minitab, SPSS, etc.
  \end{onecolentry}

  \vspace{0.2 cm}

  \begin{onecolentry}
    \textbf{OS:} Windows, Linux
  \end{onecolentry}

\end{document}